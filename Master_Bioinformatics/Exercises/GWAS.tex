\documentclass[11pt]{article}
\usepackage[a4paper,top=3cm,bottom=3cm,left=2cm,right=2cm]{geometry}
%\usepackage[authoryear,round]{natbib}
\usepackage{hyperref}
\usepackage[pdftex]{color,graphicx,epsfig}
\DeclareGraphicsRule{.pdftex}{pdf}{.pdftex}{}
\usepackage{amssymb,amsmath}
\usepackage{enumerate}
\usepackage[spanish]{babel}
\usepackage[ansinew]{inputenc}



\begin{document}


\title{\bf Genetic association studies - GWAS}
\date{}


\maketitle


\noindent \textbf{EXERCISE}: Researchers are interested in detecting new SNPs associated
with BMI (body mass index). To do so, they performed a GWAS using DNA information about 425 individuals.
Genotype information is available in plink format (files 'coronary.bed', 'coronary.bim', 'coronary.fam')
while phenotypic information can be found in the file 'coronary.txt'.


\begin{itemize}
\item Read genotypes using {\tt snpStats} library
\item Verify that both data sets are in the same order (NOTE: 'id' variable in the file 'coronary.txt' must be
used since it corresponds to the unique patient number)
\item Remove those SNPs that do not pass QC  (HWE and low MAF)
\item Assess association between BMI' and the SNPs  (NOTE: remember that you are analyzing a \textbf{quantitative trait})
\item Calculate $\lambda$ and create a Q-Q plot to assess population stratification
\item Assess association between BMI' and the SNPs adjusting for
population stratification (variables 'ev3' and 'ev4' in the file 'coronary.txt') (NOTE: remember that you are analyzing a \textbf{quantitative trait}). Are there differences with crude analysis?
\item Create a Manhattan plot (use {\tt qqman} library)
\end{itemize}



\end{document}


%%%%%%%%%%%%%%%%%%%%%%%%%%%%%%%%%%%%%%%%%%%%%%%%%%%%%%%%%%%%%%%%%%%%%%%%%%%%%%%%%%%%%%%%%%%%%%%%%%%%%%%%%%%%%%%%%%%
